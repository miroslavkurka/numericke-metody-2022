%%%%%%%%%%%%%%%%%%%%%%%%%%%%%%%%%%%%%%%%%%%%%%%%%%%%%%%%%%%%%%%%%%%%%%%%%%%
%
% Template for a LaTex article in Slovak.
%
%%%%%%%%%%%%%%%%%%%%%%%%%%%%%%%%%%%%%%%%%%%%%%%%%%%%%%%%%%%%%%%%%%%%%%%%%%%

\documentclass{article}

\usepackage[utf8]{inputenc}
\usepackage[slovak]{babel}
\usepackage[document]{ragged2e}
\usepackage{amsmath}
\usepackage{siunitx}
\usepackage{multicol}
\usepackage{textcomp}
\usepackage{amsmath, amsthm, amsfonts}
\usepackage{graphicx}
\usepackage{url}

% Theorems
%-----------------------------------------------------------------
\newtheorem{thm}{Theorem}[section]
\newtheorem{cor}[thm]{Corollary}
\newtheorem{lem}[thm]{Lemma}
\newtheorem{prop}[thm]{Proposition}
\theoremstyle{definition}
\newtheorem{defn}[thm]{Definition}
\theoremstyle{remark}
\newtheorem{rem}[thm]{Remark}

% Shortcuts.
% One can define new commands to shorten frequently used
% constructions. As an example, this defines the R and Z used
% for the real and integer numbers.
%-----------------------------------------------------------------
\def\RR{\mathbb{R}}
\def\ZZ{\mathbb{Z}}

% Similarly, one can define commands that take arguments. In this
% example we define a command for the absolute value.
% -----------------------------------------------------------------
\newcommand{\abs}[1]{\left\vert#1\right\vert}

% Operators
% New operators must defined as such to have them typeset
% correctly. As an example we define the Jacobian:
% -----------------------------------------------------------------
\DeclareMathOperator{\Jac}{Jac}

%-----------------------------------------------------------------
\title{Newtonová metóda riešenie rovníc}
\author{Miroslav Kurka\\
  \small Dept. of Biophysics\\
  \small Pavol Jozef Šafárik University in Košice\\
  \small Slovakia 
}

\begin{document}
\maketitle


\section{Úloha}

Newtonovou metódou riešte rovnicu
$$\cos{2x}-3\sin(x)= 2-4x $$
s použitím počiatočnej hodnoty $x(0) = 1$. Koľko iterácií je potrebných na dosiahnutie
presnosti na 6 desatinných miest? Výsledok porovnajte s výsledkom obdŕžaným
pomocou Octavovskej funkcie \emph{fsolve}.

\subsection{Teória}\label{sec:nothing}
Newtonovu metódu, nazývanú tiež metóda dotyčníc, môžeme použiť na riešenie rovnice $f(x) = 0$ ak je funkcia dvakrát diferencovateľná a funkcia $f(x)$ v intervale $\langle a, b \rangle$  nemení znamienko (je nenulová), pričom zároveň platí $f(a)f(b) < 0$ ($\langle a, b \rangle$ je interval separácie)\cite{Bsp}

Zostrojme dotyčnicu v arbitrárne zvolenom bode $x_0$:
$$f(x)=f(x_0)+f'(x_0)(x-x_0)$$

zaujímame sa o priesečník dotyčnice s osou x, teda môžeme dosadiť $f(x)=0$. Rovnica vyzerá nasledujúco $f(x_0)+f'(x_0)(x-x_0)=0$. Po úprave a zovšeobecnení indexov získavame rekurentý vzťah:

$$x_{n+1}=x_n-\frac{f(x_n)}{f'(x_n)}$$

Týmto spôsobom vieme aproximovať riešenie funkcie s odchyľkou závislej na počte iterácií.

\subsubsection{Algoritmus}\label{sec:nothing2}
Zadefinujeme prvú iteráciu a počiatočný bod a premennú $i$, ktorá bude držať počet iterácii. Následne cez balík \emph{symbolic} definujeme funkciu ku ktorej aproximujeme riešenie a deriváciu tejto funkcie. V cykle while uvedieme ako podmienku dostatočného riešenia: absolútna hodnota rozdielu výsledkov $x$ po sebe idúcich iteračných krokoch je väčšia ako $10^-6$. V cykle prechadzujeme $x_n$ na $x_{n+1}$ a vypočitame $x_{n+1}$. K premennej $i$ pripočítame 1. Výsledok uložíme do premennej a zmeníme typ na \emph{double}. Vypočítame tú istú rovnicu cez matlab definovanú funkciu \emph{fsolve}. 
\subsection{Výsledky}\label{sec:nothing}
Pri výpočte sa vyskytol problém dosadzovania do funkcii z balíka \emph{symbolic}, ktorý bol vyriešený premennou typu na double. Po siedmych iteráciach sa while cyklus zastavil. Dostali sme výsledok $my\_result = 1.494976$ a následne výsledok z funkcie \emph{fsolve} $matlab\_calculation\_result = 1.494976$.



\subsection{Záver}\label{sec:nothing}
Na presnosť 6 desatinných miest sa naša implematácia od nátivnej funkcie \emph{fsolve} nelíší. Je potrebných 7 iterácií aby sme dosiahli presnosti funkcie \emph{fsolve}.
% Bibliography
%-----------------------------------------------------------------
\begin{thebibliography}{99}
\bibitem{Bsp} Buša, J., Pirč, V. and Schrötter, Š. (no date) \emph{Numerické metódy, pravdepodobnosť a matematická štatistika}
\end{thebibliography}

\end{document}
