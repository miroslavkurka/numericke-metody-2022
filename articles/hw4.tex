%%%%%%%%%%%%%%%%%%%%%%%%%%%%%%%%%%%%%%%%%%%%%%%%%%%%%%%%%%%%%%%%%%%%%%%%%%%
%
% Template for a LaTex article in Slovak.
%
%%%%%%%%%%%%%%%%%%%%%%%%%%%%%%%%%%%%%%%%%%%%%%%%%%%%%%%%%%%%%%%%%%%%%%%%%%%

\documentclass{article}

\usepackage[utf8]{inputenc}
\usepackage[slovak]{babel}
\usepackage[document]{ragged2e}
\usepackage{amsmath}
\usepackage{siunitx}
\usepackage{multicol}
\usepackage{textcomp}
\usepackage{amsmath, amsthm, amsfonts}
\usepackage{graphicx}
\usepackage{url}

% Theorems
%-----------------------------------------------------------------
\newtheorem{thm}{Theorem}[section]
\newtheorem{cor}[thm]{Corollary}
\newtheorem{lem}[thm]{Lemma}
\newtheorem{prop}[thm]{Proposition}
\theoremstyle{definition}
\newtheorem{defn}[thm]{Definition}
\theoremstyle{remark}
\newtheorem{rem}[thm]{Remark}

% Shortcuts.
% One can define new commands to shorten frequently used
% constructions. As an example, this defines the R and Z used
% for the real and integer numbers.
%-----------------------------------------------------------------
\def\RR{\mathbb{R}}
\def\ZZ{\mathbb{Z}}

% Similarly, one can define commands that take arguments. In this
% example we define a command for the absolute value.
% -----------------------------------------------------------------
\newcommand{\abs}[1]{\left\vert#1\right\vert}

% Operators
% New operators must defined as such to have them typeset
% correctly. As an example we define the Jacobian:
% -----------------------------------------------------------------
\DeclareMathOperator{\Jac}{Jac}

%-----------------------------------------------------------------
\title{Newtonová metóda riešenie rovníc}
\author{Miroslav Kurka\\
  \small Dept. of Biophysics\\
  \small Pavol Jozef Šafárik University in Košice\\
  \small Slovakia 
}

\begin{document}
\maketitle


\section{Úloha}
 Vypočítajte daný integrál so zadanou presnosťou ε pomocou lichobežníkovej a Simpsonovej metódy. Koľko delení intervalu je potrebných pri jednotlivých metódach k dosiahnutiu požadovanej presnosti? Následne preveďte výpočty s využitím funkcií softvéru Octave trapz a quadv. Porovnajte aj z presným riešením.
$$\int_{0}^{5} \frac{x}{(4+x^2)} \,dx$$


\subsection{Teória}\label{sec:nothing}
Obidve metódy používajú delenie intervalu na menšie časti. Integrál potom aproximujú následovne:
\begin{itemize}
	\item Lichobežníková metóda je založená na aproximácii integrálu sčítaním lichobežníkov pod krivkou cez podintervaly.\cite{Bsp}
\item  Simpsonova metóda je založená na aproximácii integrálu za pomoci parabolickej krivky.\cite{Bsp}
\end{itemize}

\subsubsection{Lichobežníková metóda}
Lichobežníková metóda je definovaná ako 
$$ \int_{a}^{b} f(x)  dx \approx \frac{h}{2} \left( f(x_0) + 2 \sum_{i=1}^{n-1}
f(x_i) + f(x_n) \right) $$

kde $h = \frac{b-a}{n}$ je velkosť kroku a $x_0, x_1, \dots, x_{n}$ sú rovnomerne vzdialené podintervaly v intervale $[a,b]$.\cite{Bsp}
\subsection{Simpsonova metóda}
Simpsonova metóda je definovaná ako
$$ \int_{a}^{b} f(x) , dx \approx \frac{h}{3} \left( f(x_0) + 4[f(x_1) + f(x_3) + \dots+ f(x_{n-1})] +2[f(x_2) + f(x_4) + \dots+ f(x_{n-2})] + f(x_{n}) \right) $$
kde $h = \frac{b-a}{n}$ je velkosť kroku a $x_0, x_1, \dots, x_{n}$ sú rovnomerne vzdialené podintervaly v intervale $[a,b]$.\cite{Bsp}
\subsubsection{Algoritmus}

V prvom kroku inicializujeme našu funkciu, ktorú budeme aproximovať. Následne definujeme hornú, dolnú hranicu integrálu a presnosť epsilon. Zvolíme náhodné čislo počtu intervalov prvej iterácie, toto číslo musí byť násobok 2 kvôli Simpsonovej metóde. Vypočítame h a vygenerujeme pole od dolnej po hornú hranicu s krokom h. Za využitia vstavanej funkcie \emph{arrayfun} namapujeme zadanú funkciu z úlohy na body vzniknutých podintervalov. Toto vráti hodnoty funkcie v daných bodoch. Inicializujeme prvú iteráciu lichobežníkovej a Simpsonovej metódy. Pri Simpsonovej metóde v sumách používame trik s krokom indexu poľa aby sme dostali párne body a následne nepárne. Taktiež vytvoríme pomocné premenné do ktorých budeme ukladať predošlé hodnoty aproximovaných integrálov. V cyckle while, ktorý sa zastaví ak obidve metódy splnia definovanú presnosť, presunieme predošlé hodnoty integrálov do pomocných premenných. Ďalej zvýšime počet intervalov o dvojnásobok a znova prepočítame krok h. Vypočítame aproximované hodnoty integrálov. Iterujeme pokiaľ obidve metódy dosiahnu definovanú presnosť.

Výsledky a počet intervalov vypíšeme na obrazovku.
Zobrazíme výsledky z Octave funkcií \emph{trapz}, \emph{quadv} a analytické riešenie. 


\subsection{Výsledky}
V tabulke nižšie sú získane výsledky z našej implementácie vstavaných funkcií a počet potrebných intervalov. V porovnaní vidíme rozdiely v presnosti, ktoré sa avšak objavujú až pri 5. desatinom čísle.


\begin{tabular}{|l|S[round-precision=6]|}
  
  \hline
  \textbf{Hodnota} & \textbf{Výsledok} \\
  \hline
  I\_simpson & 0.99050077 \\
  I\_trapezoid & 0.99041122 \\
  number\_of\_intervals & 80 \\
  matlab\_trapezoid & 0.99041122 \\
  matlab\_simpson & 0.99050069 \\
  analytical\_solution & 0.99050073 \\
  \hline
\end{tabular}
\subsection{Záver}
Na požadovanú presnosť je potrebných 80 intervalov. Zistili sme že najbližšie riešenie k analytickému je Simpsonova metóda implementovaná cez funkciu \emph{quadv}. Lichobežníková metóda je menej presná ako Simpsonova metóda. Výsledky z Octave funkcie \emph{trapz},  sú rovnaké ako výsledky z našej vlastnej implementácie.

% Bibliography
%-----------------------------------------------------------------
\begin{thebibliography}{99}
\bibitem{Bsp} Buša, J., Pirč, V. and Schrötter, Š. (no date) \emph{Numerické metódy, pravdepodobnosť a matematická štatistika}
\end{thebibliography}

\end{document}
