\documentclass[11pt]{article}

    \usepackage[breakable]{tcolorbox}
    \usepackage{parskip} % Stop auto-indenting (to mimic markdown behaviour)
    
    \usepackage{iftex}
    \ifPDFTeX
    	\usepackage[T1]{fontenc}
    	\usepackage{mathpazo}
    \else
    	\usepackage{fontspec}
    \fi

    % Basic figure setup, for now with no caption control since it's done
    % automatically by Pandoc (which extracts ![](path) syntax from Markdown).
    \usepackage{graphicx}
    % Maintain compatibility with old templates. Remove in nbconvert 6.0
    \let\Oldincludegraphics\includegraphics
    % Ensure that by default, figures have no caption (until we provide a
    % proper Figure object with a Caption API and a way to capture that
    % in the conversion process - todo).
    \usepackage{caption}
    \DeclareCaptionFormat{nocaption}{}
    \captionsetup{format=nocaption,aboveskip=0pt,belowskip=0pt}

    \usepackage{float}
    \floatplacement{figure}{H} % forces figures to be placed at the correct location
    \usepackage{xcolor} % Allow colors to be defined
    \usepackage{enumerate} % Needed for markdown enumerations to work
    \usepackage{geometry} % Used to adjust the document margins
    \usepackage{amsmath} % Equations
    \usepackage{amssymb} % Equations
    \usepackage{textcomp} % defines textquotesingle
    % Hack from http://tex.stackexchange.com/a/47451/13684:
    \AtBeginDocument{%
        \def\PYZsq{\textquotesingle}% Upright quotes in Pygmentized code
    }
    \usepackage{upquote} % Upright quotes for verbatim code
    \usepackage{eurosym} % defines \euro
    \usepackage[mathletters]{ucs} % Extended unicode (utf-8) support
    \usepackage{fancyvrb} % verbatim replacement that allows latex
    \usepackage{grffile} % extends the file name processing of package graphics 
                         % to support a larger range
    \makeatletter % fix for old versions of grffile with XeLaTeX
    \@ifpackagelater{grffile}{2019/11/01}
    {
      % Do nothing on new versions
    }
    {
      \def\Gread@@xetex#1{%
        \IfFileExists{"\Gin@base".bb}%
        {\Gread@eps{\Gin@base.bb}}%
        {\Gread@@xetex@aux#1}%
      }
    }
    \makeatother
    \usepackage[Export]{adjustbox} % Used to constrain images to a maximum size
    \adjustboxset{max size={0.9\linewidth}{0.9\paperheight}}

    % The hyperref package gives us a pdf with properly built
    % internal navigation ('pdf bookmarks' for the table of contents,
    % internal cross-reference links, web links for URLs, etc.)
    \usepackage{hyperref}
    % The default LaTeX title has an obnoxious amount of whitespace. By default,
    % titling removes some of it. It also provides customization options.
    \usepackage{titling}
    \usepackage{longtable} % longtable support required by pandoc >1.10
    \usepackage{booktabs}  % table support for pandoc > 1.12.2
    \usepackage[inline]{enumitem} % IRkernel/repr support (it uses the enumerate* environment)
    \usepackage[normalem]{ulem} % ulem is needed to support strikethroughs (\sout)
                                % normalem makes italics be italics, not underlines
    \usepackage{mathrsfs}
    

    
    % Colors for the hyperref package
    \definecolor{urlcolor}{rgb}{0,.145,.698}
    \definecolor{linkcolor}{rgb}{.71,0.21,0.01}
    \definecolor{citecolor}{rgb}{.12,.54,.11}

    % ANSI colors
    \definecolor{ansi-black}{HTML}{3E424D}
    \definecolor{ansi-black-intense}{HTML}{282C36}
    \definecolor{ansi-red}{HTML}{E75C58}
    \definecolor{ansi-red-intense}{HTML}{B22B31}
    \definecolor{ansi-green}{HTML}{00A250}
    \definecolor{ansi-green-intense}{HTML}{007427}
    \definecolor{ansi-yellow}{HTML}{DDB62B}
    \definecolor{ansi-yellow-intense}{HTML}{B27D12}
    \definecolor{ansi-blue}{HTML}{208FFB}
    \definecolor{ansi-blue-intense}{HTML}{0065CA}
    \definecolor{ansi-magenta}{HTML}{D160C4}
    \definecolor{ansi-magenta-intense}{HTML}{A03196}
    \definecolor{ansi-cyan}{HTML}{60C6C8}
    \definecolor{ansi-cyan-intense}{HTML}{258F8F}
    \definecolor{ansi-white}{HTML}{C5C1B4}
    \definecolor{ansi-white-intense}{HTML}{A1A6B2}
    \definecolor{ansi-default-inverse-fg}{HTML}{FFFFFF}
    \definecolor{ansi-default-inverse-bg}{HTML}{000000}

    % common color for the border for error outputs.
    \definecolor{outerrorbackground}{HTML}{FFDFDF}

    % commands and environments needed by pandoc snippets
    % extracted from the output of `pandoc -s`
    \providecommand{\tightlist}{%
      \setlength{\itemsep}{0pt}\setlength{\parskip}{0pt}}
    \DefineVerbatimEnvironment{Highlighting}{Verbatim}{commandchars=\\\{\}}
    % Add ',fontsize=\small' for more characters per line
    \newenvironment{Shaded}{}{}
    \newcommand{\KeywordTok}[1]{\textcolor[rgb]{0.00,0.44,0.13}{\textbf{{#1}}}}
    \newcommand{\DataTypeTok}[1]{\textcolor[rgb]{0.56,0.13,0.00}{{#1}}}
    \newcommand{\DecValTok}[1]{\textcolor[rgb]{0.25,0.63,0.44}{{#1}}}
    \newcommand{\BaseNTok}[1]{\textcolor[rgb]{0.25,0.63,0.44}{{#1}}}
    \newcommand{\FloatTok}[1]{\textcolor[rgb]{0.25,0.63,0.44}{{#1}}}
    \newcommand{\CharTok}[1]{\textcolor[rgb]{0.25,0.44,0.63}{{#1}}}
    \newcommand{\StringTok}[1]{\textcolor[rgb]{0.25,0.44,0.63}{{#1}}}
    \newcommand{\CommentTok}[1]{\textcolor[rgb]{0.38,0.63,0.69}{\textit{{#1}}}}
    \newcommand{\OtherTok}[1]{\textcolor[rgb]{0.00,0.44,0.13}{{#1}}}
    \newcommand{\AlertTok}[1]{\textcolor[rgb]{1.00,0.00,0.00}{\textbf{{#1}}}}
    \newcommand{\FunctionTok}[1]{\textcolor[rgb]{0.02,0.16,0.49}{{#1}}}
    \newcommand{\RegionMarkerTok}[1]{{#1}}
    \newcommand{\ErrorTok}[1]{\textcolor[rgb]{1.00,0.00,0.00}{\textbf{{#1}}}}
    \newcommand{\NormalTok}[1]{{#1}}
    
    % Additional commands for more recent versions of Pandoc
    \newcommand{\ConstantTok}[1]{\textcolor[rgb]{0.53,0.00,0.00}{{#1}}}
    \newcommand{\SpecialCharTok}[1]{\textcolor[rgb]{0.25,0.44,0.63}{{#1}}}
    \newcommand{\VerbatimStringTok}[1]{\textcolor[rgb]{0.25,0.44,0.63}{{#1}}}
    \newcommand{\SpecialStringTok}[1]{\textcolor[rgb]{0.73,0.40,0.53}{{#1}}}
    \newcommand{\ImportTok}[1]{{#1}}
    \newcommand{\DocumentationTok}[1]{\textcolor[rgb]{0.73,0.13,0.13}{\textit{{#1}}}}
    \newcommand{\AnnotationTok}[1]{\textcolor[rgb]{0.38,0.63,0.69}{\textbf{\textit{{#1}}}}}
    \newcommand{\CommentVarTok}[1]{\textcolor[rgb]{0.38,0.63,0.69}{\textbf{\textit{{#1}}}}}
    \newcommand{\VariableTok}[1]{\textcolor[rgb]{0.10,0.09,0.49}{{#1}}}
    \newcommand{\ControlFlowTok}[1]{\textcolor[rgb]{0.00,0.44,0.13}{\textbf{{#1}}}}
    \newcommand{\OperatorTok}[1]{\textcolor[rgb]{0.40,0.40,0.40}{{#1}}}
    \newcommand{\BuiltInTok}[1]{{#1}}
    \newcommand{\ExtensionTok}[1]{{#1}}
    \newcommand{\PreprocessorTok}[1]{\textcolor[rgb]{0.74,0.48,0.00}{{#1}}}
    \newcommand{\AttributeTok}[1]{\textcolor[rgb]{0.49,0.56,0.16}{{#1}}}
    \newcommand{\InformationTok}[1]{\textcolor[rgb]{0.38,0.63,0.69}{\textbf{\textit{{#1}}}}}
    \newcommand{\WarningTok}[1]{\textcolor[rgb]{0.38,0.63,0.69}{\textbf{\textit{{#1}}}}}
    
    
    % Define a nice break command that doesn't care if a line doesn't already
    % exist.
    \def\br{\hspace*{\fill} \\* }
    % Math Jax compatibility definitions
    \def\gt{>}
    \def\lt{<}
    \let\Oldtex\TeX
    \let\Oldlatex\LaTeX
    \renewcommand{\TeX}{\textrm{\Oldtex}}
    \renewcommand{\LaTeX}{\textrm{\Oldlatex}}
    % Document parameters
    % Document title
    \title{Domaca Uloha 1 - LSM}
    
    
    
    \author{Miro Kurka}
    
    
    
% Pygments definitions
\makeatletter
\def\PY@reset{\let\PY@it=\relax \let\PY@bf=\relax%
    \let\PY@ul=\relax \let\PY@tc=\relax%
    \let\PY@bc=\relax \let\PY@ff=\relax}
\def\PY@tok#1{\csname PY@tok@#1\endcsname}
\def\PY@toks#1+{\ifx\relax#1\empty\else%
    \PY@tok{#1}\expandafter\PY@toks\fi}
\def\PY@do#1{\PY@bc{\PY@tc{\PY@ul{%
    \PY@it{\PY@bf{\PY@ff{#1}}}}}}}
\def\PY#1#2{\PY@reset\PY@toks#1+\relax+\PY@do{#2}}

\expandafter\def\csname PY@tok@w\endcsname{\def\PY@tc##1{\textcolor[rgb]{0.73,0.73,0.73}{##1}}}
\expandafter\def\csname PY@tok@c\endcsname{\let\PY@it=\textit\def\PY@tc##1{\textcolor[rgb]{0.25,0.50,0.50}{##1}}}
\expandafter\def\csname PY@tok@cp\endcsname{\def\PY@tc##1{\textcolor[rgb]{0.74,0.48,0.00}{##1}}}
\expandafter\def\csname PY@tok@k\endcsname{\let\PY@bf=\textbf\def\PY@tc##1{\textcolor[rgb]{0.00,0.50,0.00}{##1}}}
\expandafter\def\csname PY@tok@kp\endcsname{\def\PY@tc##1{\textcolor[rgb]{0.00,0.50,0.00}{##1}}}
\expandafter\def\csname PY@tok@kt\endcsname{\def\PY@tc##1{\textcolor[rgb]{0.69,0.00,0.25}{##1}}}
\expandafter\def\csname PY@tok@o\endcsname{\def\PY@tc##1{\textcolor[rgb]{0.40,0.40,0.40}{##1}}}
\expandafter\def\csname PY@tok@ow\endcsname{\let\PY@bf=\textbf\def\PY@tc##1{\textcolor[rgb]{0.67,0.13,1.00}{##1}}}
\expandafter\def\csname PY@tok@nb\endcsname{\def\PY@tc##1{\textcolor[rgb]{0.00,0.50,0.00}{##1}}}
\expandafter\def\csname PY@tok@nf\endcsname{\def\PY@tc##1{\textcolor[rgb]{0.00,0.00,1.00}{##1}}}
\expandafter\def\csname PY@tok@nc\endcsname{\let\PY@bf=\textbf\def\PY@tc##1{\textcolor[rgb]{0.00,0.00,1.00}{##1}}}
\expandafter\def\csname PY@tok@nn\endcsname{\let\PY@bf=\textbf\def\PY@tc##1{\textcolor[rgb]{0.00,0.00,1.00}{##1}}}
\expandafter\def\csname PY@tok@ne\endcsname{\let\PY@bf=\textbf\def\PY@tc##1{\textcolor[rgb]{0.82,0.25,0.23}{##1}}}
\expandafter\def\csname PY@tok@nv\endcsname{\def\PY@tc##1{\textcolor[rgb]{0.10,0.09,0.49}{##1}}}
\expandafter\def\csname PY@tok@no\endcsname{\def\PY@tc##1{\textcolor[rgb]{0.53,0.00,0.00}{##1}}}
\expandafter\def\csname PY@tok@nl\endcsname{\def\PY@tc##1{\textcolor[rgb]{0.63,0.63,0.00}{##1}}}
\expandafter\def\csname PY@tok@ni\endcsname{\let\PY@bf=\textbf\def\PY@tc##1{\textcolor[rgb]{0.60,0.60,0.60}{##1}}}
\expandafter\def\csname PY@tok@na\endcsname{\def\PY@tc##1{\textcolor[rgb]{0.49,0.56,0.16}{##1}}}
\expandafter\def\csname PY@tok@nt\endcsname{\let\PY@bf=\textbf\def\PY@tc##1{\textcolor[rgb]{0.00,0.50,0.00}{##1}}}
\expandafter\def\csname PY@tok@nd\endcsname{\def\PY@tc##1{\textcolor[rgb]{0.67,0.13,1.00}{##1}}}
\expandafter\def\csname PY@tok@s\endcsname{\def\PY@tc##1{\textcolor[rgb]{0.73,0.13,0.13}{##1}}}
\expandafter\def\csname PY@tok@sd\endcsname{\let\PY@it=\textit\def\PY@tc##1{\textcolor[rgb]{0.73,0.13,0.13}{##1}}}
\expandafter\def\csname PY@tok@si\endcsname{\let\PY@bf=\textbf\def\PY@tc##1{\textcolor[rgb]{0.73,0.40,0.53}{##1}}}
\expandafter\def\csname PY@tok@se\endcsname{\let\PY@bf=\textbf\def\PY@tc##1{\textcolor[rgb]{0.73,0.40,0.13}{##1}}}
\expandafter\def\csname PY@tok@sr\endcsname{\def\PY@tc##1{\textcolor[rgb]{0.73,0.40,0.53}{##1}}}
\expandafter\def\csname PY@tok@ss\endcsname{\def\PY@tc##1{\textcolor[rgb]{0.10,0.09,0.49}{##1}}}
\expandafter\def\csname PY@tok@sx\endcsname{\def\PY@tc##1{\textcolor[rgb]{0.00,0.50,0.00}{##1}}}
\expandafter\def\csname PY@tok@m\endcsname{\def\PY@tc##1{\textcolor[rgb]{0.40,0.40,0.40}{##1}}}
\expandafter\def\csname PY@tok@gh\endcsname{\let\PY@bf=\textbf\def\PY@tc##1{\textcolor[rgb]{0.00,0.00,0.50}{##1}}}
\expandafter\def\csname PY@tok@gu\endcsname{\let\PY@bf=\textbf\def\PY@tc##1{\textcolor[rgb]{0.50,0.00,0.50}{##1}}}
\expandafter\def\csname PY@tok@gd\endcsname{\def\PY@tc##1{\textcolor[rgb]{0.63,0.00,0.00}{##1}}}
\expandafter\def\csname PY@tok@gi\endcsname{\def\PY@tc##1{\textcolor[rgb]{0.00,0.63,0.00}{##1}}}
\expandafter\def\csname PY@tok@gr\endcsname{\def\PY@tc##1{\textcolor[rgb]{1.00,0.00,0.00}{##1}}}
\expandafter\def\csname PY@tok@ge\endcsname{\let\PY@it=\textit}
\expandafter\def\csname PY@tok@gs\endcsname{\let\PY@bf=\textbf}
\expandafter\def\csname PY@tok@gp\endcsname{\let\PY@bf=\textbf\def\PY@tc##1{\textcolor[rgb]{0.00,0.00,0.50}{##1}}}
\expandafter\def\csname PY@tok@go\endcsname{\def\PY@tc##1{\textcolor[rgb]{0.53,0.53,0.53}{##1}}}
\expandafter\def\csname PY@tok@gt\endcsname{\def\PY@tc##1{\textcolor[rgb]{0.00,0.27,0.87}{##1}}}
\expandafter\def\csname PY@tok@err\endcsname{\def\PY@bc##1{\setlength{\fboxsep}{0pt}\fcolorbox[rgb]{1.00,0.00,0.00}{1,1,1}{\strut ##1}}}
\expandafter\def\csname PY@tok@kc\endcsname{\let\PY@bf=\textbf\def\PY@tc##1{\textcolor[rgb]{0.00,0.50,0.00}{##1}}}
\expandafter\def\csname PY@tok@kd\endcsname{\let\PY@bf=\textbf\def\PY@tc##1{\textcolor[rgb]{0.00,0.50,0.00}{##1}}}
\expandafter\def\csname PY@tok@kn\endcsname{\let\PY@bf=\textbf\def\PY@tc##1{\textcolor[rgb]{0.00,0.50,0.00}{##1}}}
\expandafter\def\csname PY@tok@kr\endcsname{\let\PY@bf=\textbf\def\PY@tc##1{\textcolor[rgb]{0.00,0.50,0.00}{##1}}}
\expandafter\def\csname PY@tok@bp\endcsname{\def\PY@tc##1{\textcolor[rgb]{0.00,0.50,0.00}{##1}}}
\expandafter\def\csname PY@tok@fm\endcsname{\def\PY@tc##1{\textcolor[rgb]{0.00,0.00,1.00}{##1}}}
\expandafter\def\csname PY@tok@vc\endcsname{\def\PY@tc##1{\textcolor[rgb]{0.10,0.09,0.49}{##1}}}
\expandafter\def\csname PY@tok@vg\endcsname{\def\PY@tc##1{\textcolor[rgb]{0.10,0.09,0.49}{##1}}}
\expandafter\def\csname PY@tok@vi\endcsname{\def\PY@tc##1{\textcolor[rgb]{0.10,0.09,0.49}{##1}}}
\expandafter\def\csname PY@tok@vm\endcsname{\def\PY@tc##1{\textcolor[rgb]{0.10,0.09,0.49}{##1}}}
\expandafter\def\csname PY@tok@sa\endcsname{\def\PY@tc##1{\textcolor[rgb]{0.73,0.13,0.13}{##1}}}
\expandafter\def\csname PY@tok@sb\endcsname{\def\PY@tc##1{\textcolor[rgb]{0.73,0.13,0.13}{##1}}}
\expandafter\def\csname PY@tok@sc\endcsname{\def\PY@tc##1{\textcolor[rgb]{0.73,0.13,0.13}{##1}}}
\expandafter\def\csname PY@tok@dl\endcsname{\def\PY@tc##1{\textcolor[rgb]{0.73,0.13,0.13}{##1}}}
\expandafter\def\csname PY@tok@s2\endcsname{\def\PY@tc##1{\textcolor[rgb]{0.73,0.13,0.13}{##1}}}
\expandafter\def\csname PY@tok@sh\endcsname{\def\PY@tc##1{\textcolor[rgb]{0.73,0.13,0.13}{##1}}}
\expandafter\def\csname PY@tok@s1\endcsname{\def\PY@tc##1{\textcolor[rgb]{0.73,0.13,0.13}{##1}}}
\expandafter\def\csname PY@tok@mb\endcsname{\def\PY@tc##1{\textcolor[rgb]{0.40,0.40,0.40}{##1}}}
\expandafter\def\csname PY@tok@mf\endcsname{\def\PY@tc##1{\textcolor[rgb]{0.40,0.40,0.40}{##1}}}
\expandafter\def\csname PY@tok@mh\endcsname{\def\PY@tc##1{\textcolor[rgb]{0.40,0.40,0.40}{##1}}}
\expandafter\def\csname PY@tok@mi\endcsname{\def\PY@tc##1{\textcolor[rgb]{0.40,0.40,0.40}{##1}}}
\expandafter\def\csname PY@tok@il\endcsname{\def\PY@tc##1{\textcolor[rgb]{0.40,0.40,0.40}{##1}}}
\expandafter\def\csname PY@tok@mo\endcsname{\def\PY@tc##1{\textcolor[rgb]{0.40,0.40,0.40}{##1}}}
\expandafter\def\csname PY@tok@ch\endcsname{\let\PY@it=\textit\def\PY@tc##1{\textcolor[rgb]{0.25,0.50,0.50}{##1}}}
\expandafter\def\csname PY@tok@cm\endcsname{\let\PY@it=\textit\def\PY@tc##1{\textcolor[rgb]{0.25,0.50,0.50}{##1}}}
\expandafter\def\csname PY@tok@cpf\endcsname{\let\PY@it=\textit\def\PY@tc##1{\textcolor[rgb]{0.25,0.50,0.50}{##1}}}
\expandafter\def\csname PY@tok@c1\endcsname{\let\PY@it=\textit\def\PY@tc##1{\textcolor[rgb]{0.25,0.50,0.50}{##1}}}
\expandafter\def\csname PY@tok@cs\endcsname{\let\PY@it=\textit\def\PY@tc##1{\textcolor[rgb]{0.25,0.50,0.50}{##1}}}

\def\PYZbs{\char`\\}
\def\PYZus{\char`\_}
\def\PYZob{\char`\{}
\def\PYZcb{\char`\}}
\def\PYZca{\char`\^}
\def\PYZam{\char`\&}
\def\PYZlt{\char`\<}
\def\PYZgt{\char`\>}
\def\PYZsh{\char`\#}
\def\PYZpc{\char`\%}
\def\PYZdl{\char`\$}
\def\PYZhy{\char`\-}
\def\PYZsq{\char`\'}
\def\PYZdq{\char`\"}
\def\PYZti{\char`\~}
% for compatibility with earlier versions
\def\PYZat{@}
\def\PYZlb{[}
\def\PYZrb{]}
\makeatother


    % For linebreaks inside Verbatim environment from package fancyvrb. 
    \makeatletter
        \newbox\Wrappedcontinuationbox 
        \newbox\Wrappedvisiblespacebox 
        \newcommand*\Wrappedvisiblespace {\textcolor{red}{\textvisiblespace}} 
        \newcommand*\Wrappedcontinuationsymbol {\textcolor{red}{\llap{\tiny$\m@th\hookrightarrow$}}} 
        \newcommand*\Wrappedcontinuationindent {3ex } 
        \newcommand*\Wrappedafterbreak {\kern\Wrappedcontinuationindent\copy\Wrappedcontinuationbox} 
        % Take advantage of the already applied Pygments mark-up to insert 
        % potential linebreaks for TeX processing. 
        %        {, <, #, %, $, ' and ": go to next line. 
        %        _, }, ^, &, >, - and ~: stay at end of broken line. 
        % Use of \textquotesingle for straight quote. 
        \newcommand*\Wrappedbreaksatspecials {% 
            \def\PYGZus{\discretionary{\char`\_}{\Wrappedafterbreak}{\char`\_}}% 
            \def\PYGZob{\discretionary{}{\Wrappedafterbreak\char`\{}{\char`\{}}% 
            \def\PYGZcb{\discretionary{\char`\}}{\Wrappedafterbreak}{\char`\}}}% 
            \def\PYGZca{\discretionary{\char`\^}{\Wrappedafterbreak}{\char`\^}}% 
            \def\PYGZam{\discretionary{\char`\&}{\Wrappedafterbreak}{\char`\&}}% 
            \def\PYGZlt{\discretionary{}{\Wrappedafterbreak\char`\<}{\char`\<}}% 
            \def\PYGZgt{\discretionary{\char`\>}{\Wrappedafterbreak}{\char`\>}}% 
            \def\PYGZsh{\discretionary{}{\Wrappedafterbreak\char`\#}{\char`\#}}% 
            \def\PYGZpc{\discretionary{}{\Wrappedafterbreak\char`\%}{\char`\%}}% 
            \def\PYGZdl{\discretionary{}{\Wrappedafterbreak\char`\$}{\char`\$}}% 
            \def\PYGZhy{\discretionary{\char`\-}{\Wrappedafterbreak}{\char`\-}}% 
            \def\PYGZsq{\discretionary{}{\Wrappedafterbreak\textquotesingle}{\textquotesingle}}% 
            \def\PYGZdq{\discretionary{}{\Wrappedafterbreak\char`\"}{\char`\"}}% 
            \def\PYGZti{\discretionary{\char`\~}{\Wrappedafterbreak}{\char`\~}}% 
        } 
        % Some characters . , ; ? ! / are not pygmentized. 
        % This macro makes them "active" and they will insert potential linebreaks 
        \newcommand*\Wrappedbreaksatpunct {% 
            \lccode`\~`\.\lowercase{\def~}{\discretionary{\hbox{\char`\.}}{\Wrappedafterbreak}{\hbox{\char`\.}}}% 
            \lccode`\~`\,\lowercase{\def~}{\discretionary{\hbox{\char`\,}}{\Wrappedafterbreak}{\hbox{\char`\,}}}% 
            \lccode`\~`\;\lowercase{\def~}{\discretionary{\hbox{\char`\;}}{\Wrappedafterbreak}{\hbox{\char`\;}}}% 
            \lccode`\~`\:\lowercase{\def~}{\discretionary{\hbox{\char`\:}}{\Wrappedafterbreak}{\hbox{\char`\:}}}% 
            \lccode`\~`\?\lowercase{\def~}{\discretionary{\hbox{\char`\?}}{\Wrappedafterbreak}{\hbox{\char`\?}}}% 
            \lccode`\~`\!\lowercase{\def~}{\discretionary{\hbox{\char`\!}}{\Wrappedafterbreak}{\hbox{\char`\!}}}% 
            \lccode`\~`\/\lowercase{\def~}{\discretionary{\hbox{\char`\/}}{\Wrappedafterbreak}{\hbox{\char`\/}}}% 
            \catcode`\.\active
            \catcode`\,\active 
            \catcode`\;\active
            \catcode`\:\active
            \catcode`\?\active
            \catcode`\!\active
            \catcode`\/\active 
            \lccode`\~`\~ 	
        }
    \makeatother

    \let\OriginalVerbatim=\Verbatim
    \makeatletter
    \renewcommand{\Verbatim}[1][1]{%
        %\parskip\z@skip
        \sbox\Wrappedcontinuationbox {\Wrappedcontinuationsymbol}%
        \sbox\Wrappedvisiblespacebox {\FV@SetupFont\Wrappedvisiblespace}%
        \def\FancyVerbFormatLine ##1{\hsize\linewidth
            \vtop{\raggedright\hyphenpenalty\z@\exhyphenpenalty\z@
                \doublehyphendemerits\z@\finalhyphendemerits\z@
                \strut ##1\strut}%
        }%
        % If the linebreak is at a space, the latter will be displayed as visible
        % space at end of first line, and a continuation symbol starts next line.
        % Stretch/shrink are however usually zero for typewriter font.
        \def\FV@Space {%
            \nobreak\hskip\z@ plus\fontdimen3\font minus\fontdimen4\font
            \discretionary{\copy\Wrappedvisiblespacebox}{\Wrappedafterbreak}
            {\kern\fontdimen2\font}%
        }%
        
        % Allow breaks at special characters using \PYG... macros.
        \Wrappedbreaksatspecials
        % Breaks at punctuation characters . , ; ? ! and / need catcode=\active 	
        \OriginalVerbatim[#1,codes*=\Wrappedbreaksatpunct]%
    }
    \makeatother

    % Exact colors from NB
    \definecolor{incolor}{HTML}{303F9F}
    \definecolor{outcolor}{HTML}{D84315}
    \definecolor{cellborder}{HTML}{CFCFCF}
    \definecolor{cellbackground}{HTML}{F7F7F7}
    
    % prompt
    \makeatletter
    \newcommand{\boxspacing}{\kern\kvtcb@left@rule\kern\kvtcb@boxsep}
    \makeatother
    \newcommand{\prompt}[4]{
        {\ttfamily\llap{{\color{#2}[#3]:\hspace{3pt}#4}}\vspace{-\baselineskip}}
    }
    

    
    % Prevent overflowing lines due to hard-to-break entities
    \sloppy 
    % Setup hyperref package
    \hypersetup{
      breaklinks=true,  % so long urls are correctly broken across lines
      colorlinks=true,
      urlcolor=urlcolor,
      linkcolor=linkcolor,
      citecolor=citecolor,
      }
    % Slightly bigger margins than the latex defaults
    
    \geometry{verbose,tmargin=1in,bmargin=1in,lmargin=1in,rmargin=1in}
    
    

\begin{document}
    
    \maketitle
    
    

    
    \hypertarget{uxfaloha-1}{%
\section{Úloha 1}\label{uxfaloha-1}}

\begin{itemize}
\item
  Majme tabuľku nameraných dát. Nájdite polynóm stupňa n ≤ n\_max, ktorý
  najlepšie aproximuje funkčnú závislosť y(t) v zmysle (neváženej)
  metódy najmenších štvorcov. Zobrazte závislosť rezíduí, t.j. súčet
  štvorcov, na stupni polynómu pre všetky skúmané polynómy.
\item
  Porovnajte výsledky s výsledkami obdŕžanými využitím zabudovanej
  fitovacej funkcie polyfit
\end{itemize}

Data:

\begin{Shaded}
\begin{Highlighting}[]
\NormalTok{t\_i }\OperatorTok{=}\NormalTok{ [}\DecValTok{2} \FloatTok{2.5} \DecValTok{3} \FloatTok{3.5} \DecValTok{4} \FloatTok{4.5} \DecValTok{5}\NormalTok{]}
\NormalTok{y\_i }\OperatorTok{=}\NormalTok{ [}\FloatTok{7.14} \FloatTok{6.56} \FloatTok{5.98} \FloatTok{5.55} \FloatTok{5.71} \FloatTok{6.01} \FloatTok{6.53}\NormalTok{]}
\end{Highlighting}
\end{Shaded}

    \hypertarget{teoretickuxfd-uxfavod}{%
\subsection{Teoretický úvod}\label{teoretickuxfd-uxfavod}}

Cielom LS metódy je dosiahnúť takú aproximačnú funkciu, ktorá sa
najbližšie modeluje namerané dáta. Narozdiel napríklad od interpolácie
splajnami, pri LS nie je potrebné aby funkcia prechádzala nameraními
bodmy, metóda je založená na minimalizacií súčtu vzdialenosti štvorcou
medzi skutočnou hodnotou a aproximačnou funkciou.

Uveďme príklad ak máme dva namerané body\footnote{V skutočnosti je
  nezmyselné robiť aproximáciu z dvoch bodov kvôli podmienenosti ale pre
  n bodov už \(\LaTeX\)-ove rovnice sú prilíš zdlhavé na písanie.}

Zvoľme aproximačnú funkciu \(\varphi(x_i)=a_0 + a_1x\). Teraz odčítajme
hodnotu \(\varphi(x_i)\) od skutočnej nameranej hodnoty \(y_i\).
Dostávame \(a_0 + a_1x - y_i\), vezmime štvorec tejto hodnoty
\((a_0 + a_1x_i - y_i)^{2}\), to spravíme pre každú hodnotu. Následne
sčítame takto získané hodnoty, chceme aby suma mocnín vzdialeností
(residue) bola čo najmenšia \(R=\sum_{i=0}^{n}(a_0 + a_1x_i -y_i)^{2}\).
Máme definovanú minimalizačnú úlohu, pre lokálne extrémy platí, že
derivácia je rovná nule. Najprv zderivujeme podľa \(a_0\) a upravíme:
\[2\sum_{i=0}^{n}(a_0 + a_1x_i -y_i)=0\]
\[2(a_0n + a_1\sum_{i=0}^{n}x_i -\sum_{i=0}^{n}y_i)=0\]
\[a_0n + a_1\sum_{i=0}^{n}x_i =\sum_{i=0}^{n}y_i\] Nasledne zderivujeme
podla \(a_1\): \[2\sum_{i=0}^{n}(a_0 + a_1x_i -y_i)x_i=0\]
\[2(a_0\sum_{i=0}^{n}x_i + a_1\sum_{i=0}^{n}x_i^{2} -\sum_{i=0}^{n}y_ix_i)=0\]
\[a_0\sum_{i=0}^{n}x_i + a_1\sum_{i=0}^{n}x_i^{2}=\sum_{i=0}^{n}y_ix_i\]
Dostávame sústavu rovníc, môžeme zapísať v maticovom tvare:
\[\begin{pmatrix} n & \sum_{i=0}^{n}x_i  \\ \sum_{i=0}^{n}x_i & \sum_{i=0}^{n}x_i^{2}   \end{pmatrix}\begin{pmatrix} a_0 \\ a_1 \end{pmatrix}=\begin{pmatrix} \sum_{i=0}^{n}y_i \\ \sum_{i=0}^{n}y_ix_i\end{pmatrix} \]

Všeobecne platí:

\[\begin{pmatrix}  n &   \dots &\sum_{i=0}^{n}x_i^{k} \\  \vdots & \ddots & \\ \sum_{i=0}^{n}x_i^{k} &        & \sum_{i=0}^{n}x_i^{2k}  \end{pmatrix} \begin{pmatrix}  a_0  \\ \vdots \\ a_{i} \end{pmatrix}=\begin{pmatrix}  \sum_{i=0}^{n}y_i \\ \vdots \\ \sum_{i=0}^{n}y_ix_i^{k} \end{pmatrix}\]

Dostávame maticovú rovnicu v tvare \(\mathbf{V}a= \mathbf{Y}\), kde
riešením dostávame a vektor, kde elementy sú hľadané najmenšie konštanty
pre aproximačný polynóm.

\hypertarget{popis-programu}{%
\subsubsection{Popis programu}\label{popis-programu}}

    \begin{tcolorbox}[breakable, size=fbox, boxrule=1pt, pad at break*=1mm,colback=cellbackground, colframe=cellborder]
\prompt{In}{incolor}{2}{\boxspacing}
\begin{Verbatim}[commandchars=\\\{\}]
\PY{c}{\PYZsh{}declares plotting env for jupyter octave kernel}
\PY{n+nb}{graphics\PYZus{}toolkit} \PY{p}{(}\PY{l+s}{\PYZdq{}qt\PYZdq{}}\PY{p}{)}\PY{p}{;}
\end{Verbatim}
\end{tcolorbox}

    \begin{Verbatim}[commandchars=\\\{\}]

    \end{Verbatim}

    \begin{tcolorbox}[breakable, size=fbox, boxrule=1pt, pad at break*=1mm,colback=cellbackground, colframe=cellborder]
\prompt{In}{incolor}{10}{\boxspacing}
\begin{Verbatim}[commandchars=\\\{\}]
\PY{c}{\PYZsh{}Released under MIT License }
\PY{c}{\PYZsh{}Copyright (c) 2022 Miroslav Kurka}

\PY{n}{t\PYZus{}i} \PY{p}{=} \PY{p}{[}\PY{l+m+mi}{2} \PY{l+m+mf}{2.5} \PY{l+m+mi}{3} \PY{l+m+mf}{3.5} \PY{l+m+mi}{4} \PY{l+m+mf}{4.5} \PY{l+m+mi}{5}\PY{p}{]}\PY{p}{;} \PY{c}{\PYZsh{} t\PYZus{}i is our x\PYZus{}i value}
\PY{n}{y\PYZus{}i} \PY{p}{=} \PY{p}{[}\PY{l+m+mf}{7.14} \PY{l+m+mf}{6.56} \PY{l+m+mf}{5.98} \PY{l+m+mf}{5.55} \PY{l+m+mf}{5.71} \PY{l+m+mf}{6.01} \PY{l+m+mf}{6.53}\PY{p}{]}\PY{p}{;}
\PY{n}{residues}\PY{p}{=}\PY{p}{[}\PY{p}{]}\PY{p}{;}
\PY{c}{\PYZsh{}n= length(t\PYZus{}i) }
\PY{n}{n}\PY{p}{=}\PY{n+nb}{columns}\PY{p}{(}\PY{n}{t\PYZus{}i}\PY{p}{)}\PY{p}{;}
\PY{n}{k}\PY{p}{=} \PY{n}{n}\PY{o}{\PYZhy{}}\PY{l+m+mi}{1}\PY{p}{;} \PY{c}{\PYZsh{} degree of the polynomial}
\PY{c}{\PYZsh{} init matrices}
\PY{n}{V}\PY{p}{=}\PY{n+nb}{zeros}\PY{p}{(}\PY{n}{n}\PY{p}{,}\PY{n}{n}\PY{p}{)}\PY{p}{;}
\PY{n}{Y}\PY{p}{=}\PY{n+nb}{zeros}\PY{p}{(}\PY{n}{n}\PY{p}{,}\PY{l+m+mi}{1}\PY{p}{)}\PY{p}{;}


    
\PY{c}{\PYZsh{} fill Y matrix}

\PY{k}{for} \PY{n}{i}\PY{p}{=}\PY{l+m+mi}{1}\PY{p}{:}\PY{n}{n}
   \PY{n}{Y}\PY{p}{(}\PY{n}{i}\PY{p}{)}\PY{p}{=}\PY{n+nb}{sum}\PY{p}{(}\PY{n}{y\PYZus{}i}\PY{o}{.*}\PY{p}{(}\PY{n}{t\PYZus{}i}\PY{o}{.\PYZca{}}\PY{p}{(}\PY{n}{i}\PY{o}{\PYZhy{}}\PY{l+m+mi}{1}\PY{p}{)}\PY{p}{)}\PY{p}{)}\PY{p}{;} \PY{c}{\PYZsh{} i\PYZhy{}1 so we can get t\PYZus{}i\PYZca{}0 = 1, hence the n not k in the for loop range}
\PY{k}{endfor}

\PY{c}{\PYZsh{} fill the V matrix}
    
\PY{k}{for} \PY{n}{i}\PY{p}{=}\PY{l+m+mi}{1}\PY{p}{:}\PY{n}{n}
  \PY{k}{for} \PY{n}{j}\PY{p}{=}\PY{l+m+mi}{1}\PY{p}{:}\PY{n}{n}
    \PY{k}{if} \PY{n}{i}\PY{o}{==}\PY{l+m+mi}{1} \PY{o}{\PYZam{}\PYZam{}} \PY{n}{j}\PY{o}{==}\PY{l+m+mi}{1}
      \PY{n}{V}\PY{p}{(}\PY{n}{i}\PY{p}{,}\PY{n}{j}\PY{p}{)}\PY{p}{=}\PY{n}{k}\PY{p}{;}
    \PY{k}{else}
      \PY{n}{V}\PY{p}{(}\PY{n}{i}\PY{p}{,}\PY{n}{j}\PY{p}{)}\PY{p}{=}\PY{n+nb}{sum}\PY{p}{(}\PY{n}{t\PYZus{}i}\PY{o}{.\PYZca{}}\PY{p}{(}\PY{n}{i}\PY{o}{+}\PY{n}{j}\PY{p}{)}\PY{o}{\PYZhy{}}\PY{l+m+mi}{2}\PY{p}{)}\PY{p}{;}
    \PY{k}{endif}
    \PY{k}{endfor}
\PY{k}{endfor}
    

\PY{n}{st}\PY{p}{=}\PY{l+m+mi}{7}\PY{p}{;}
\PY{c}{\PYZsh{} get each degree polynomial by slicing the V }
\PY{k}{for} \PY{n}{pos}\PY{p}{=}\PY{l+m+mi}{1}\PY{p}{:}\PY{n}{st}
    \PY{k}{if} \PY{n}{st}\PY{o}{\PYZhy{}}\PY{n}{pos}\PY{o}{==}\PY{l+m+mi}{0}\PY{p}{,} \PY{k}{break}\PY{p}{;} \PY{k}{end} \PY{c}{\PYZsh{} end for loop since 0x0 doesnt exist, and indexing is nightmare in matlab}
    \PY{n}{V}\PY{p}{=}\PY{n}{V}\PY{p}{(}\PY{l+m+mi}{1}\PY{p}{:}\PY{n}{st}\PY{o}{\PYZhy{}}\PY{n}{pos}\PY{p}{,}\PY{l+m+mi}{1}\PY{p}{:}\PY{n}{st}\PY{o}{\PYZhy{}}\PY{n}{pos}\PY{p}{)}\PY{p}{;} \PY{c}{\PYZsh{} reverse slicing, we reduce matrix from to 6x6 to 5x5 ...}
    \PY{n}{Y}\PY{p}{=}\PY{n}{Y}\PY{p}{(}\PY{l+m+mi}{1}\PY{p}{:}\PY{n}{st}\PY{o}{\PYZhy{}}\PY{n}{pos}\PY{p}{)}\PY{p}{;}
    \PY{n}{a\PYZus{}coefficients}\PY{p}{=}\PY{n}{V}\PY{o}{\PYZbs{}}\PY{n}{Y}\PY{p}{;}
    \PY{n}{lst\PYZus{}func}\PY{p}{=}\PY{n+nb}{polyval}\PY{p}{(}\PY{n}{flip}\PY{p}{(}\PY{n}{a\PYZus{}coefficients}\PY{p}{)}\PY{p}{,}\PY{n}{y\PYZus{}i}\PY{p}{)}\PY{p}{;}
    \PY{n}{residues}\PY{p}{(}\PY{n}{pos}\PY{p}{)}\PY{p}{=} \PY{n+nb}{sum}\PY{p}{(}\PY{p}{(}\PY{n}{y\PYZus{}i}\PY{o}{\PYZhy{}}\PY{n}{lst\PYZus{}func}\PY{p}{)}\PY{o}{.\PYZca{}}\PY{l+m+mi}{2}\PY{p}{)}\PY{p}{;}
\PY{k}{endfor}
\PY{c}{\PYZsh{} flip the residues since by slicing we got 6x6 ie k=6 first }
\PY{n}{residues}\PY{p}{=}\PY{n}{flip}\PY{p}{(}\PY{n}{residues}\PY{p}{)}\PY{p}{;}
\PY{n}{poly\PYZus{}degrees}\PY{p}{=}\PY{p}{(}\PY{l+m+mi}{1}\PY{p}{:}\PY{n}{k}\PY{p}{)}\PY{p}{;}
\PY{n+nb}{semilogy}\PY{p}{(}\PY{n}{poly\PYZus{}degrees}\PY{p}{,}\PY{n}{residues}\PY{p}{)}
\PY{n+nb}{grid} \PY{n}{on}
\PY{n+nb}{hold} \PY{n}{on}


\PY{c}{\PYZsh{} task b) compare with polyfit }

\PY{n}{matlab\PYZus{}residues}\PY{p}{=}\PY{p}{[}\PY{p}{]}\PY{p}{;}

\PY{k}{for} \PY{n}{i}\PY{p}{=}\PY{l+m+mi}{1}\PY{p}{:}\PY{l+m+mi}{6}
    \PY{n}{a\PYZus{}matlab\PYZus{}coefficients}\PY{p}{=}\PY{n+nb}{polyfit}\PY{p}{(}\PY{n}{t\PYZus{}i}\PY{p}{,}\PY{n}{y\PYZus{}i}\PY{p}{,}\PY{n}{i}\PY{p}{)}\PY{p}{;}
    \PY{n}{mat\PYZus{}func}\PY{p}{=}\PY{n+nb}{polyval}\PY{p}{(}\PY{n}{a\PYZus{}matlab\PYZus{}coefficients}\PY{p}{,}\PY{n}{y\PYZus{}i}\PY{p}{)}\PY{p}{;}
    \PY{n}{matlab\PYZus{}residues}\PY{p}{(}\PY{n}{i}\PY{p}{)}\PY{p}{=}\PY{n+nb}{sum}\PY{p}{(}\PY{p}{(}\PY{n}{y\PYZus{}i}\PY{o}{\PYZhy{}}\PY{n}{mat\PYZus{}func}\PY{p}{)}\PY{o}{.\PYZca{}}\PY{l+m+mi}{2}\PY{p}{)}\PY{p}{;}
\PY{k}{endfor}

\PY{n+nb}{semilogy}\PY{p}{(}\PY{n}{poly\PYZus{}degrees}\PY{p}{,} \PY{n}{matlab\PYZus{}residues}\PY{p}{,}\PY{l+s}{\PYZdq{}red\PYZdq{}}\PY{p}{)}
\PY{n+nb}{legend}\PY{p}{(}\PY{l+s}{\PYZdq{}Our Method\PYZdq{}}\PY{p}{,}\PY{l+s}{\PYZdq{}MATLAB polyfit\PYZdq{}}\PY{p}{)}
\PY{n+nb}{xlabel}\PY{p}{(}\PY{l+s}{\PYZdq{}Degrees of polynomial\PYZdq{}}\PY{p}{)}
\PY{n+nb}{ylabel}\PY{p}{(}\PY{l+s}{\PYZdq{}Residue\PYZdq{}}\PY{p}{)}
\PY{n+nb}{hold} \PY{n}{off}
\end{Verbatim}
\end{tcolorbox}

    \begin{Verbatim}[commandchars=\\\{\}]







    \end{Verbatim}

    \begin{center}
    \adjustimage{max size={0.9\linewidth}{0.9\paperheight}}{output_3_1.png}
    \end{center}
    { \hspace*{\fill} \\}
    
    \hypertarget{zuxe1ver}{%
\subsubsection{Záver}\label{zuxe1ver}}

Z grafu je viditeľné, že najlepší ``fit'' je pri polynóme so stupnňom 5
v oboch prípadoch. Taktiež, je zretelné, že implementovaná funkcia v
MATLAB knižnici je presnejšia. Dôvodom je najskôr optimalnejšia
implementácia algoritmu LSM alebo výber iného ``fitovacieho'' algoritmu.


    % Add a bibliography block to the postdoc
    
    
    
\end{document}
